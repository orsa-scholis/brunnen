\documentclass{article}
\usepackage[utf8]{inputenc}

\title{Open-Source Software}
\author{Lukas Bischof \& Philipp Fehr }
\date{März 2020}

\usepackage{natbib}
\usepackage{graphicx}

\begin{document}

\maketitle

\section{Abstract}
Dieses Dokument soll eine einfache Übersicht bieten, warum das EvaWeB-Tool unter einer OpenSource Lizenz entwickelt wurde.

\section{Warum OpenSource}
Das EvaWeB-Tool wurde auf Wunsch der zwei Entwickler \footnote{Lukas Bischof und Philipp Fehr} unter einer OpenSource Lizenz entwickelt.

\subsection{Vorteile}
\begin{enumerate}
    \item Quellofen, d.h. jeder \footnote{Internetbenutzer} kann den Quellcode einsehen.
    \begin{enumerate}
        \item Dadurch ist die Chance, dass grosse Sicherheits- oder Leistungsschwachstellen aufgedeckt werden viel grösser.
        \item Personen, die an der Software interessiert können einfach zusätzliche Funktionalitäten entwickeln und der ganzen OpenSource Gemeinschaft wieder zur Verfügung stellen.
    \end{enumerate}
\end{enumerate}

\section{OpenSource und EvaWeB}

\subsection{Lizenz}
Für das EvaWeB Projekt wurde die AGPL \citep{agpl} Lizenz verwendet.
Diese Lizenz verhindert die kommerzielle Nutzung durch Dritte.

\renewcommand{\bibsection}{\section{Quellen}}
\bibliographystyle{plain}
\bibliography{references}
\end{document}
